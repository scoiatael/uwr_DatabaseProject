
% Podstawowe definicje dla wszystkich dokumentów

\documentclass[12pt]{mwart}
%\setlength{\textwidth}{83pt}

\usepackage[OT4,plmath]{polski}
\usepackage{amsmath,amssymb,amsfonts,amsthm,mathtools}
\usepackage{color}
\usepackage{fontspec}
\usepackage{listings,times}

\usepackage{bbm}
\usepackage[colorlinks=false]{hyperref}
\usepackage{url}
\usepackage{graphicx}

\graphicspath{{images/}}

\newcommand{\HRule}{\rule{\linewidth}{0.5mm}}

\newcommand{\term}[1]{
  \indent\textbf{#1}
  \vspace{5pt}
}

\usepackage{multicol}

\usepackage{capt-of}

\usepackage{lmodern} \normalfont
%\DeclareFontShape{EU1}{ptm}{bx}{n} { <-> ssub * cmr/bx/n }{}
%\DeclareFontShape{EU1}{ptm}{m}{sc} { <-> ssub * cmr/m/sc }{}

%\usepackage{fancyhdr}
%\pagestyle{fancy}
%\fancyhead{}
%\fancyhead[LE,LO]{Iron Coach. \doctitle}
%\fancyhead[RO,RE]{\rightmark}
%\fancyfoot{}
%\fancyfoot[LE,RO]{}
%\fancyfoot[CO,CE]{\thepage}
%\addtolength{\headheight}{1.5pt}

\widowpenalty=10000
\clubpenalty=10000
\linepenalty=1000
\hyphenpenalty=10000
\raggedbottom

\renewcommand{\labelitemi}{}
\renewcommand{\labelitemii}{}
\renewcommand{\labelitemiii}{}
\renewcommand{\labelitemiv}{}

\newcommand{\titlep}[2] {
  \newcommand{\doctitle}{#1}
  \thispagestyle{empty}
  \begin{titlepage}
    \begin{center}
      \textsc{Studencka Pracownia Baz Danych}\\[0.5cm]
      \textsc{\large Instytut Informatyki\\Uniwersytetu Wrocławskiego}\\[1.5cm]


      \vspace{3.5cm}

      % Author and supervisor
      \begin{minipage}{\textwidth}
        \begin{center} \large
          Łukasz \textsc{Czapliński}
        \end{center}
      \end{minipage}

      \vspace{0.5cm}



      % Title
      \HRule \\[0.4cm]
      { \Large Dokumentacja projektu \\[0.5cm] }
      { \Huge \bfseries Coffee Shop  \\[1cm] }

      \textsc{\Large #1\\\large Wersja #2}\\[0.5cm]

      \HRule \\[1.5cm]

      \vspace{1cm}

      
      \vfill
      

      \vspace{1cm}

      % Bottom of the page
      {\large Wrocław 2014}

    \end{center}
  \end{titlepage}
  \clearpage
  \setcounter{page}{2}
}

\newcommand{\chist}[1]{
  \begin{table}
  \centering
  \caption{Historia zmian}
  \begin{tabular}{l | l | p{5cm} | p{5cm}}
    Wersja & Data & Opis & Autor \\ 
    \hline
    \noalign{\smallskip}
    #1
  \end{tabular}
\end{table}
}
