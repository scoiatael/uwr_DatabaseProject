
% Podstawowe definicje dla wszystkich dokumentów

\documentclass[12pt]{mwart}
%\setlength{\textwidth}{83pt}

\usepackage[OT4,plmath]{polski}
\usepackage{amsmath,amssymb,amsfonts,amsthm,mathtools}
\usepackage{color}
\usepackage{fontspec}
\usepackage{listings,times}

\usepackage{bbm}
\usepackage[colorlinks=false]{hyperref}
\usepackage{url}
\usepackage{graphicx}

\graphicspath{{images/}}

\newcommand{\HRule}{\rule{\linewidth}{0.5mm}}

\newcommand{\term}[1]{
  \indent\textbf{#1}
  \vspace{5pt}
}

\usepackage{multicol}

\usepackage{capt-of}

\usepackage{lmodern} \normalfont
%\DeclareFontShape{EU1}{ptm}{bx}{n} { <-> ssub * cmr/bx/n }{}
%\DeclareFontShape{EU1}{ptm}{m}{sc} { <-> ssub * cmr/m/sc }{}

%\usepackage{fancyhdr}
%\pagestyle{fancy}
%\fancyhead{}
%\fancyhead[LE,LO]{Iron Coach. \doctitle}
%\fancyhead[RO,RE]{\rightmark}
%\fancyfoot{}
%\fancyfoot[LE,RO]{}
%\fancyfoot[CO,CE]{\thepage}
%\addtolength{\headheight}{1.5pt}

\widowpenalty=10000
\clubpenalty=10000
\linepenalty=1000
\hyphenpenalty=10000
\raggedbottom

\renewcommand{\labelitemi}{}
\renewcommand{\labelitemii}{}
\renewcommand{\labelitemiii}{}
\renewcommand{\labelitemiv}{}

\newcommand{\titlep}[2] {
  \newcommand{\doctitle}{#1}
  \thispagestyle{empty}
  \begin{titlepage}
    \begin{center}
      \textsc{Studencka Pracownia Baz Danych}\\[0.5cm]
      \textsc{\large Instytut Informatyki\\Uniwersytetu Wrocławskiego}\\[1.5cm]


      \vspace{3.5cm}

      % Author and supervisor
      \begin{minipage}{\textwidth}
        \begin{center} \large
          Łukasz \textsc{Czapliński}
        \end{center}
      \end{minipage}

      \vspace{0.5cm}



      % Title
      \HRule \\[0.4cm]
      { \Large Dokumentacja projektu \\[0.5cm] }
      { \Huge \bfseries Iron Coach  \\[1cm] }

      \textsc{\Large #1\\\large Wersja #2}\\[0.5cm]

      \HRule \\[1.5cm]

      \vspace{1cm}

      
      \vfill
      

      \vspace{1cm}

      % Bottom of the page
      {\large Wrocław 2014}

    \end{center}
  \end{titlepage}
  \clearpage
  \setcounter{page}{2}
}

\newcommand{\chist}[1]{
  \begin{table}
  \centering
  \caption{Historia zmian}
  \begin{tabular}{l | l | p{5cm} | p{5cm}}
    Wersja & Data & Opis & Autor \\ 
    \hline
    \noalign{\smallskip}
    #1
  \end{tabular}
\end{table}
}


\begin{document}
\titlep{Dokumentacja techniczna}{1.0}
\chist{1.0 & 2014-06-13 & Powstanie dokumentu & Łukasz Czapliński\\
        }
\tableofcontents
\clearpage
\section{Wprowadzenie}
\subsection{Cel dokumentu}
Dokument ten ma na celu omówienie struktury, instalacji oraz sposobów zarządzania aplikacji bazodanowej Coffee Shop.
\section{System}
Na aplikację CoffeeShop skłądają się 2 komponenty: baza danych napędzana przez silnik PostgreSQL oraz graficzny interfejs użytkownika (GUI) napisany w Haskellu z użyciem bibliotek HDBC, VTY oraz VTY-UI. \\
Baza danych ma na celu gromadzenie i przetwarzanie informacji, natomiast  GUI ma pozwalać na łatwe wprowadzanie danych oraz ich przeglądanie.\\
Aplikacja Coffee Shop została stworzona z myślą o systemach Linuksowych i uruchamianiu jej przez terminal. Nie gwarantuje się jej działania na żadnym systemie. Testowana była na systemie ArchLinux64.
\section{Instalacja}
W celu rozpoczęcia użytkowania aplikacji Coffee Shop należy:\begin{enumerate}
  \item Pobrać, zainstalować oraz skonfigurować PostgreSQL. 
  \item Uruchomić PostgreSQL.
  \item Stworzyć bazę danych o nazwie 'projekt.db' oraz upewnić się, że użytkownik uruchamiający GUI będzie miał prawo zalogować się do tej bazy jako użytkownik Postgres bez podawania hasła: zywkle można to sprawdzić wydając na tym użytkowniku polecenie \textit{psql -U postgres projekt.db} i sprawdzić, czy zostanie poprawnie wykonane.
  \item Sprawdzić, czy dołączona do dystrubucji wersja GUI działa na danym systemie (jest skompilowana dla Linuxa x64). Jeśli nie, to należy ją skompilować na tym systemie poleceniem \textit{make bin} $\rightarrow$  patrz niżej.
  \item Uruchomić GUI, zalogować się jako \textbf{admin} i wybrać opcję 'Stwórz bazę'.
  \item (opcjonalnie) Zapełnić bazę przykładowymi danymi (polecenie \textit{make example}).
\end{enumerate}
W celu zbudowania aplikacji ze źródeł naleźy posiadać narzędzia:\begin{itemize}
  \item[\textit{sh}] = 4.3.18(1)-release
  \item[\textit{make}] = 4.0 (GNU Make)
  \item[\textit{cabal-install}] = 1.20.0.2
  \item[\textit{GHC}] = 7.8.2
\end{itemize}
Prawdopodbnie będzie wymagało to instalacji dodatkowych bibliotek dla Haskella. Można to zrobić poleceniem \textit{cabal update \&\& cabal install hdbc hdbc-postgresql vty-4.7.5 vty-ui}.
\section{Zarządzanie}
GUI pozwala na łatwe przeglądanie oraz dodawanie użytkowników. Pozwala także na wyczyszczenie bazy. Nalezy zalogować się jako \textbf{admin} i wybrać odpowiednią opcję. \\
Każdy użytkownik (klient/dostawca/właściciel) powinien znać swój numer ID. Można go poznać przeglądając jako admin listę użytkowników.
\end{document}
